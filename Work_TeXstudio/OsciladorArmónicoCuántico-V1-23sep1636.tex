\documentclass[11pt,a4paper,spanish]{book}
\usepackage{estilo_unir-1}



%---------------------------
%título del trabajo y autor
%---------------------------
\title{El oscilador armónico cuántico}
\author{Edgar Rodríguez García and Jordan Vicaña Alburqueque and Benghir Tello Pacheco and Marvin Jiménez Agüero}
\date{13 de octubre de 2025}
\profesor{Rodrigo Gil-Merino y Rubio}

%---------------------------
%marges
%---------------------------
%\usepackage[margin=1.9cm]{geometry}
%---------------------------
%---------------------------
%---------------------------
%---------------------------
\begin{document}
\renewcommand{\listfigurename}{Índice de figuras}
\renewcommand{\listtablename}{Índice de tablas}
\renewcommand{\contentsname}{Índice de contenidos}
\renewcommand{\figurename}{Figura}
\renewcommand{\tablename}{Tabla} 

\maketitle
\frontmatter
\tableofcontents
\listoffigures %atención, quitar si no hay figuras!!!
\listoftables %atención, quitar si no hay tablas!!!

\chapter{Resumen}
Aquí se introducirá un breve resumen en español del trabajo realizado (extensión máxima: 150 palabras). Este resumen debe incluir el objetivo o propósito de la investigación, la metodología, los resultados y las conclusiones (obviamente, todo muy resumido, pero así se sabe de un vistazo lo que se va uno a encontrar a continuación).\\

JORDAN-----Resumen

MARVIN-----Resumen

BENHIR-----Resumen

EDGAR------Resumen


{\bf Palabras clave:} se deben incluir de 3 a 5 palabras claves en español (pueden ser conceptos formados por más de una palabra)


\chapter{Abstract}
Aquí debe introducirse la versión en {\bf inglés} del Resumen anterior.\\


{\bf Keywords:} se deben incluir de 3 a 5 palabras claves en inglés  (pueden ser conceptos formados por más de una palabra)




\mainmatter
\chapter{Introducción}

En la Introducción se debe resumir de forma esquemática pero suficientemente clara lo esencial de cada una de las partes del trabajo. La lectura de esta parte debe contextualizar perfectamente todo el trabajo y debe estar PLAGADA DE REFERENCIAS.\\

Las referencias NO están para rellenar. Son un TRIBUTO a las personas que hicieron en primer lugar una investigación o aportaron una idea, por tanto, se deben citar LOS TRABAJOS ORIGINALES DE LOS AUTORES, y no un libro de texto donde he visto que hablan de algo.\\

Es una parte muy importante de la memoria. Las ideas principales a transmitir son la identificación del problema a tratar, la justificación de su importancia, los objetivos generales a grandes rasgos y un adelanto de la contribución que esperas hacer.\\

A modo de guía, la Introducción debe contener estos tres apartados:
\begin{itemize}
\item Motivación / justificación del tema a tratar
\item Planteamiento del Trabajo
\item Estructura del Trabajo
\end{itemize}


ATENCIÓN:  Si queremos citar a alguien, por ejemplo porque vamos a hablar de Latex \citep{lamport1994} o porque, según las ideas de \cite{ackerman2017}, la liga de fútbol inglesa debe tener torneos de desempate, pues tenemos que hacerlo correctamente.

JORDAN-----Introduccion

MARVIN-----Introduccion

BENHIR-----Introduccion

EDGAR------Introduccion


\chapter{Contexto y estado de la cuestión}\label{contexto}

En esta Sección~\ref{contexto} debemos demostrar que conocemos lo que se ha hecho en el ámbito que estamos desarrollando el Trabajo. En nuestro caso, que se ha buscado la bibliografía y referencias suficientes y que esas ideas se han volcado en el Trabajo en la línea de los objetivos que perseguimos o que queremos transmitir.

JORDAN-----Contexto y estado

MARVIN-----Contexto y estado

BENHIR-----Contexto y estado

EDGAR------Contexto y estado


\chapter{Objetivos}

Esquematizar claramente los objetivos del Trabajo, las ideas que queremos demostrar. Podemos tener varios tipos de objetivos. Deben ordenarse claramente.

\chapter{Oscilador armónico cuántico}

Aquí desarrollaremos nuestro Trabajo. Contrastaremos la ideas entre varios autores y, si es posible, con las nuestras. Podemos incluir los subapartados que necesitemos.


\section{Qué es el oscilador en física}
Desarrollo de tema....
\subsection{Por qué interesa el oscilador en la ciencia y tecnología}
Desarrollo de tema....


\section{Oscilador armónico clásico}
Desarrollo de tema....
\subsection{La energía en el movimiento cinético y potencial}
Desarrollo de tema....

\section{Dualidad onda - partícula}
Desarrollo de tema....
\subsection{Principio de incertidumbre}
Desarrollo de tema....

\section{Cuantización}
Desarrollo de tema....
\subsection{Max Planck. Cuantos(niveles) de energía}
Desarrollo de tema....
\subsection{Werner Heisenberg, Max Born, Pascual Jordan.  uso de matrices para representar posición y momento}
Desarrollo de tema....


\chapter{Conclusiones}

Las Conclusiones es otra parte muy IMPORTANTE de la memoria. Deben ser muy clara. Si es posible se pueden itemizar o, mejor, poner un párrafo por idea con un pequeño título ilustrativo.

JORDAN-----Conclusiones

MARVIN-----Conclusiones

BENHIR-----Conclusiones

EDGAR------Conclusiones



\addcontentsline{toc}{chapter}{Bibliografía}
\begin{thebibliography}{a}
%%\bibitem{etiqueta} \textsc{Autores},\textit{nombre referencia.}Información addicional
\bibitem[Lamport(1994)]{lamport1994} Lamport, L. (1994) \emph{\LaTeX: a document preparation system}, Addison
Wesley, Massachusetts, 2nd ed.
\bibitem[Ackerman(2017)]{ackerman2017} Ackerman, E. (2017) Why the English Premier League Should Have Playoffs.  Balls.ie. 
\end{thebibliography}
%\bibliographystyle{plain} 
%\bibliography{bibliografia}


\appendix
\chapter{Apéndices}

Aquí se pueden poner desarrollos matemáticos engorrosos de los que se puede prescindir en el cuerpo principal de la memoria u otros añadidos que aportan información pero no encajan correctamente en las secciones anteriores.\\

Si no ya apéndices, quitar esta Sección

\end{document}





















